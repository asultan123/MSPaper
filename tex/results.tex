
\chapter{HERO Architecture Simulation}
\label{chap:results}

% so lets say we have a concrete instance of hero-t and we want to assess its
% performance in a cycle accurate simulator with real layers from a network. We do
% this with a platform simulator that has a front end that takes a mode library
% and produces latency, utilization and access cost results.  

\section{Simulation Enviornment}
\label{chap:hero:sim_platform}

% Sim runs a CIGAR pass over the input model library if stats dict not available.
% It then performs a layer equivelence pass that converts an unsupported
% convolution layer to an equivelent GEMM using balanced lowering. Overhead Of
% lowering and lifting is automatically added to the overall latency of the cycle
% accurate simulation. Simulation only targets conv layers. 

\begin{figure}[ht]
    \centering
    \includegraphics[scale=0.58]{fig/hero-sim-frontend.pdf}
    \caption{Hardware Implementation Taxonomy adapted from \cite{maestro}}
    \label{fig:hw_taxonomy}
\end{figure}

\begin{figure}[ht]
    \centering
    \includegraphics[scale=0.58]{fig/hero-sim-backend.pdf}
    \caption{Hardware Implementation Taxonomy adapted from \cite{maestro}}
    \label{fig:hw_taxonomy}
\end{figure}

% \section{SystemC Model}
% \label{chap:hero:sim_platform:sysc_side}

% SystemC models takes in arch config. Arch config usually comes out of a tempo.
% Currently systemc model only considers horizontal mapping of kernel loops. All
% muxing/ interconnect logic handled with how individual banks are accessed in the
% architecture simulation loops. After systemc backend recieves sim config. It
% instantiates an arch with the required config. It generates an equivelent layer
% following dimensions of layer sent in from front end and then it generates the
% descriptor program based on the method discussed in (REFERENCE SAM SUBSECTION)
% Finally the backend performs a cycle accurate simulation of the instantiated
% architecture. After simulation concludes, ofmap memory contents are scanned to
% validate output and if output is valid, latency, access counts and utilization
% statistics are returned to the platform frontend for high level analysis. Model
% is simulated in isolation. DRAM access counts are estimated in the backend but
% no actual simulation of DRAM occurs.



\section{Experimental Results}
\label{chap:hero:sim_platform:cigar_side}

Prominent model's convolution layers were are assessed using the simulation platform  
namely resnet-50 mobilenetv3 and vgg16. Performance for each of the three conrete
architectures suggested by tempo is reported below.

\subsection{Latency and Speedup over CPU Baseline}
\label{chap:hero:sim_platform:cigar_side}

\subsection{Energy}
\label{chap:hero:sim_platform:cigar_side}

\subsection{Utilization}
\label{chap:hero:sim_platform:cigar_side}

\subsection{Per network results}
\label{chap:hero:sim_platform:cigar_side}

\subsection{DRAM Bandwidth}
\label{chap:hero:sim_platform:cigar_side}

\subsection{Descriptor program scaling}
\label{chap:hero:sim_platform:cigar_side}

\subsection{Area}
\label{chap:conv_gemm_equiv:overhead}

% Resources
